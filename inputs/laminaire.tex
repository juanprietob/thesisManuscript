\newpage
\ifodd \thepage ~\newpage ~\fi

\pagestyle{empty}
%\addcontentsline{toc}{subsubsection}{}

\begin{changemargin}{-2cm}{-1cm}

\textbf{Abstract} \\

Geometric modeling has been one of the most researched areas 
in the medical domain.
Today, there is not a well established methodology to model the shape of an organ. 
There are many approaches available and each one of them have different strengths and weaknesses.

Most state of the art methods to model shape use surface information only.
There is an increasing need for techniques to support volumetric information.
Besides shape characterization, a technique to differentiate objects by shape is needed.
This requires computing statistics on shape.

%The current challenge of research in life sciences is to create
%models to represent the surface, the interior of an object,
%and give statistical differences based on shape.

In this work, we use a technique for shape modeling that is
able to model surface and internal features and 
is suited to compute shape statistics.

Using this technique, a procedure to model the human cerebral cortex is proposed. 
This novel representation offers new possibilities to analyze cortical lesions
and compute shape statistics on the cortex.

The second part of this work proposes a methodology to 
parameterize the interior of an object.
The method is flexible and can 
enhance the visual or the physical properties of an object.

The geometric modeling enhanced with physical parameters is used 
to produce simulated magnetic resonance images.
This image simulation approach is validated by analyzing the behavior and performance of classic segmentation algorithms for real images.\\

\textbf{Keywords}\\
S-rep, texture synthesis, shape statistiques, medical image simulation, cortex modeling.\\
\\

\textbf{R\'esum\'e} \\

La mod\'elisation g\'eom\'etrique a \'et\'e l'un des sujets les plus \'etudi\'es pour la repr\'esentation des structures anatomiques dans le domaine m\'edical.
Aujourd'hui, il n'y a toujours pas de m\'ethode bien \'etablie pour mod\'eliser la forme d'un organe.
Cependant, il y a plusieurs types d'approches disponibles et chaque approche a ses forces et ses faiblesses.

La plupart des m\'ethodes de pointe utilisent uniquement l'information surfacique mais
un besoin croissant de mod\'eliser l'information volumique des objets appara\^{\i}t.
En plus de la description g\'eom\'etrique, il faut pouvoir diff\'erencier les objets d'une population selon leur forme.
Cela n\'ecessite de disposer des statistiques sur la forme dans organe dans une population donn\'e.

Dans ce travail de th\`ese, on utilise une repr\'esentation capable de mod\'eliser les caract\'eristiques surfaciques et internes d'un objet.
La repr\'esentation choisie (s-rep) a en plus l'avantage de permettre de d\'eterminer les statistiques de forme pour une population d'objets.

En s'appuyant sur cette repr\'esentation, une proc\'edure pour mod\'eliser le cortex c\'er\'ebral humain est propos\'ee.
Cette nouvelle mod\'elisation offre de nouvelles possibilit\'es pour analyser les l\'esions corticales
et calculer des statistiques de forme sur le cortex.

La deuxi\`eme partie de ce travail propose une m\'ethodologie pour d\'ecrire de mani\`ere param\'etrique l'int\'erieur d'un objet.
La m\'ethode est flexible et peut am\'eliorer l'aspect visuel ou la description des propri\'et\'es physiques d'un objet.

La mod\'elisation g\'eom\'etrique enrichie avec des param\`etres physiques volumiques
est utilis\'ee pour la simulation d'image par r\'esonance magn\'etique pour produire des simulations plus r\'ealistes.
Cette approche de simulation d'images est valid\'ee en analysant le comportement et les performances des m\'ethodes de segmentations
classiquement utilis\'ees pour traiter des images r\'eelles du cerveau.


\end{changemargin}
