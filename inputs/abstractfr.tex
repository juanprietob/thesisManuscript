
% ********** Abstract **********
\newpage
\section*{\huge Résumé}

La modélisation géométrique a été l'un des sujets les plus étudiés pour la représentation des structures anatomiques dans le domaine médical.
Aujourd'hui, il n'y a toujours pas de méthode bien établie pour modéliser la forme d'un organe.
Cependant, il y a plusieurs types d'approches disponibles et chaque approche a ses forces et ses faiblesses.

La plupart des méthodes de pointe utilisent uniquement l'information surfacique mais
un besoin croissant de modéliser l'information volumique des objets apparaît.
En plus de la description géométrique, il faut pouvoir différencier les objets d'une population selon leur forme.
Cela nécessite de disposer des statistiques sur la forme dans organe dans une population donné.

Dans ce travail de thèse, on utilise une représentation capable de modéliser les caractéristiques surfaciques et internes d'un objet.
La représentation choisie (s-rep) a en plus l'avantage de permettre de déterminer les statistiques de forme pour une population d'objets.

En s'appuyant sur cette représentation, une procédure pour modéliser le cortex cérébral humain est proposée.
Cette nouvelle modélisation offre de nouvelles possibilités pour analyser les lésions corticales
et calculer des statistiques de forme sur le cortex.

La deuxième partie de ce travail propose une méthodologie pour décrire de manière paramétrique l'intérieur d'un objet.
La méthode est flexible et peut améliorer l'aspect visuel ou la description des propriétés physiques d'un objet.

La modélisation géométrique enrichie avec des paramètres physiques volumiques
est utilisée pour la simulation d'image par résonance magnétique pour produire des simulations plus réalistes.
Cette approche de simulation d'images est validée en analysant le comportement et les performances des méthodes de segmentations
classiquement utilisées pour traiter des images réelles du cerveau.