\chapter{Conclusions and perspectives}
%\addcontentsline{toc}{part}{Conclusions and perspectives}
\label{chapter:Conclusions}

\section{Synopsis of the work}

The principal objective of this dissertation was to propose a technique able 
to model internal and external features of organs. 
Describing internal features and 
surface characteristics closes the gap 
between virtual objects and objects in the real world. 
Objects with such characteristics are needed to improve 
existent simulation procedures. 

After reviewing the state of the art techniques for 
shape modeling, it can be concluded that 
s-reps provide features superior to other techniques. 
One of the most important characteristic of this technique
is the possibility to describe the interior of an object. 
This quality was widely used through this dissertation.
It allows including information from different spatial scales
or sources such as histology images and/or physical parameters. 

To prove the capabilities of s-reps, 
the human brain became the target organ to be modeled.
From this task came the 
most important contribution of this dissertation,
which is the procedure to model the 
cerebral cortex using s-reps. 

The cerebral cortex has never been modeled 
with techniques describing internal features. 
From now on, a single non-branching object 
can be used to describe both cortical surfaces plus the interior. 
S-reps provide new ways to analyze cortical columns and analyze cortical lesions in a different parametric space.

Besides describing the shape of the cortex, it was proven 
that a population of cortices can be described by a mean shape, a set of eigenmodes 
and some coefficients. CPNS and the cortex representation 
enabled a compact representation or a statistical description of thickness 
variations in the cortex. 

The second objective of this dissertation, \textit{i.e.}, 
to provide a mechanism to model the volumetric properties of an object,
was fulfilled with a method to synthesize solids. 
This technique is compatible with s-reps and 
allows describing the interior of the object using parameters from various sources such as: 
RGB textured images; images from histology; images from $\mu MR$ and $\mu CT$; and physical parameters. 

A simulation of the brain was done using 
the techniques just described. 
S-reps of the subcortical structures were fitted 
to real MRI images segmented with \textit{Freesurfer}.
The subcortical structures include: amygdala, caudate, hippocampus, lateral ventricles, pallidum and putamen.

Using this set of s-reps, 
the internal properties of the objects were enhanced 
with textures of physical parameters. 
MRI images were simulated using the virtual object and \textit{SimuBloch}. 
The MRIs were validated using the automatic segmentation 
procedure proposed by \textit{Freesurfer}.

\section{Future work}

There are three possibilities identified for future work.
Improvement of the statistical shape description of the cortex, 
improvement of the simulation framework
and working towards the establishment of a technique 
to estimate scalar features based on shape descriptions.

The shape description of the cortex 
with s-reps and CPNS could
provide new research opportunities to
analyze shape variability in control and pathological subjects for crosswise (multiple subjects)
and longitudinal (same subject but multiple acquisitions) scenarios.
The shape description of the cortex could help researchers understand 
brain development and to identify the principal shape variations as the human brain grows. 
%Enhancing current knowledge should lead to novel theories. 

The simulation framework can be improved 
by acquiring better textured samples of physical parameters 
and generating statistical shape descriptions of the organ structures 
modeled with s-reps.

As mentioned in Section \ref{sec:ConclusionsSimu} the full potential of 
the approach to simulate images is not achieved. The textured samples 
do not contain structural information and they have a low resolution. 

With the statistical shape descriptions of organs, 
a completely different set of structures can be generated
using a few coefficients.
Control and pathological shapes can be combined to produce new simulated images.

There are two major applications for this framework, 
the optimization of image acquisition parameters 
and the validation of segmentation algorithms.

Finally, 
a methodology to estimate scalar features based on shape description needs to 
be established. 
For example, we wish to estimate the gray matter anisotropies
in thalamic nuclei based on a fitted s-rep to the structure
and clinical resolution DTI images. 

The nuclei are only visible in a high resolution DTI acquisition. 
Unfortunately, this acquisition cannot be done routinely.
In order to identify this feature, 
a set of patients with DTI and $T1-$weighted image
acquisitions will be used in the study. 

The procedure will use the $T1-$weighted images to fit s-reps.
The fitting produces tight representations of an s-rep
and increases the correspondence between internal positions $[u, v, \tau]$ among thalami.

Shape statistics will be computed using CPNS, 
and a statistical analysis will be done for 
the nuclei density using the coordinate system $[u, v, \tau]$. 
This is possible since the internal positions are in correspondence.

The statistics will contain shape and nuclei density 
descriptions for each position inside the s-rep. 

To estimate the nuclei density in a new thalamus, 
the mean shape produced by CPNS is fitted to the structure.
The nuclei density can be estimated based on 
the shape deformation and the density statistical description. 

This approach could be use to detect other type of scalar features. 
