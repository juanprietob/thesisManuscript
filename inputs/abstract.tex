% ********** Abstract **********
\cleardoublepage

\section*{\huge Abstract}
\addcontentsline{toc}{chapter}{Abstract}

Geometric modeling has been one of the most researched areas 
in the medical domain.
Today, there is not a well established methodology to model the shape of an organ. 
There are many approaches available and each one of them have different strengths and weaknesses.

Most state of the art methods to model shape use surface information only.
There is an increasing need for techniques to support volumetric information.
Besides shape characterization, a technique to differentiate objects by shape is needed.
This requires computing statistics on shape.

%The current challenge of research in life sciences is to create
%models to represent the surface, the interior of an object,
%and give statistical differences based on shape.

In this work, we use a technique for shape modeling that is
able to model surface and internal features, and 
is suited to compute shape statistics.

Using this technique (s-rep), a procedure to model the human cerebral cortex is proposed. 
This novel representation offers new possibilities to analyze cortical lesions
and compute shape statistics on the cortex.

The second part of this work proposes a methodology to 
parameterize the interior of an object.
The method is flexible and can 
enhance the visual aspect or the description of physical properties of an object.

The geometric modeling enhanced with physical parameters is used 
to produce simulated magnetic resonance images.
This image simulation approach is validated by analyzing the behavior and performance of classic segmentation algorithms for real images.
