
\graphicspath{{Introduction/images/}}

\chapter{Introduction}
\label{chapter:Introduction}

Recent advances on digital imaging modalities
and computation have enabled an emerging field 
studying the biological variability of human anatomy. 
This field is called CA (computational anatomy).

CA as defined by \cite{Grenander1998} has three principal aspects: construction of
anatomical manifolds or shape modeling by points, curves, surfaces and volumes;
comparison of these models; and statistical analysis of shape variability.

The statistics allows for inference and hypothesis testing of disease states.
This is done using the statistical description
to determine the probability of a hypothesis to be true or false.
In other words, a new sample is classified as diseased or healthy 
according to its shape.

The automated construction of anatomical models 
has been the main interest of many research groups around the world. 
These models are based on algorithms and equations that capture the behavior
and/or the appearance of the object.

Besides hypothesis testing, the models are also used for 
various types of simulation procedures such as simulation of medical images.
Medical imaging consists in acquiring details of the interior of the human body by using different techniques. 
Among these techniques we find MRI (magnetic resonance imaging), CT (computed tomography),
US (ultrasound) or PET (positron emission tomography). 
They are widely used to diagnose and plan treatment for patients with the information acquired in the image.
These imaging devices can be very expensive, they require regular maintenance and are used only by specialized staff. 
In this context, image simulation is defined as the process of producing a synthesized image 
from a specific modality using the geometry of a virtual object. 
The virtual object in question could be an organ or a system of organs.

One purpose of image simulation is to improve imaging devices.
This is done by acquiring better understanding of image acquisition phenomena
or calibrating a specific set of parameters that will produce a desired outcome 
and then translate the experience to the real device.
Another use for simulated images is to evaluate the performance of segmentation algorithms as 
the results can be directly correlated to the virtual model.

Other types of simulations are done to understand biological process,
muscle deformation, cardiac cycle, respiratory cycle, cortical folding etc. 

Knowing all the potential applications on the medical domain,
the current challenge today is to create a virtual human, in other words,
to create a model capable of integrating anatomical, physiological, mechanical,  
biological and physical information. 
This virtual human could be used to improve simulation procedures
and also hypothesis testing for disease states. 

The first objective of this dissertation 
is to provide a technique generic enough to model
the shape of various organs, 
giving the possibility to include information 
from different spatial scales, imaging modalities or 
other sources.
To model the shape of an object, there are many approaches available.
Notably deformable models have proven to be successful. 
They are able to automate, to a certain degree, 
image segmentation of structures and they also produce statistics on the deformations. 
Unfortunately, some of the techniques 
don't provide the necessary mechanisms for further use on simulation procedures. 
Chapter \ref{chapter:3DModels} summarizes the important features of the techniques 
frequently used to model the geometry of objects.
The choice of the modeling technique should meet some specific requirements 
that will be used to model the human cerebral cortex. 
The representation should support the extremely 
folded geometry of the cortex
and also be capable to locate 
internal features or 
provide a sense of orientation at every place inside the object. 

Chapter \ref{chapter:cortexModeling} explains 
a methodology to create a cortex representation. 
The modeling technique relates to 
the true shape of the cortex which is that of a folded slab
as most of the state of the art methods only use surface information or 
model the folds locally. 

The second objective 
is to provide a mechanism
to synthesize solids, meaning 
surface and internal properties 
of the object must be available.
The procedure to synthesize solids uses small two dimensions
textured images as exemplars. The synthesized result is visually and statistically 
similar to the exemplar.
The textured images can be from various sources 
including images from the web, 
histology and complex ones such as 
MRI physical parameters acquired with relaxometry techniques.
The generated solids can be used
to enhance the visualization and/or internal properties of the models created
using the techniques described in Chapter \ref{chapter:3DModels}.
Chapter \ref{chapter:textureSynthesis} explains the solid synthesis procedure.

The third objective is to use
the geometric description enhanced 
with internal properties such as physical parameters
and generate simulated MRIs.
Chapter \ref{chapter:MRISimulation} explains MRI simulation and 
uses the techniques for geometric modeling and
solid synthesis to create some virtual objects.
These objects are used to simulate MRIs, which are
validated with well known segmentation algorithms. 

Chapter \ref{chapter:Conclusions} concludes the current research study 
and future perspectives are declared. 


\newpage