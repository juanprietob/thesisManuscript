\subsection{Introduction}
\label{sec:intro}

Medical imaging consists in acquiring details of the interior of the human body by using different techniques. 
Among these techniques we find MRI (magnetic resonance imaging), CT (computed tomography),
US (ultra sounds) or PET (positron emission tomography). They are widely used to diagnose, plan and treat patients with the information acquired in the image.
Nowadays, it is desired to achieve higher resolution levels in the acquisitions,
therefore experimental imaging such as $\mu{MRI}, \mu{CT}$ or $SR \mu{CT}$ (Synchrotron Radiation X-Rays Computed Micro-Tomography) 
are still evolving. Unfortunately, they are not used in vivo because of problems related to dose radiation.
$SR \mu{CT}$ is able to produce images up to 0.28 $\mu{m}$, presenting
unique characteristics in terms of spatial density resolution and signal to noise ratio.
Consequently, it is the reference tool to investigate the micro structure of bone samples \cite{revol2002}.

These kinds of experimental imaging devices are very expensive, they require regular maintenance and are used only by specialized staff. 
The produced data is very limited since in some occasions is private, thus making it unavailable to other research purposes, 
or the acquisition protocols are performed sparingly. Image simulation is an alternative to produce low cost datasets of an image modality. 
There are two basic approaches. The first one consists in 
applying the physics present in the acquisition process to a digital model \cite{CHAR-09}.
The digital model can be used afterwards to validate and improve segmentation or quantification algorithms.

The second approach is related to texture synthesis, it was intended to mimic mammograms \cite{Castella:08} or
to designing scaffolds based on the bone micro structure \cite{DBLP:conf/smi/HoldsteinFPB09}. %Designing scaffolds represents a growing interest 
%in bio-materials, specially in tissue engineering, as they are used to support the stem cell structure 
%and help the process of tissue reconstruction 

In this paper, we propose a method related to the second approach, which is generic enough to reproduce 3D $\mu{MR}$ and $SR \mu{CT}$ images. 
The texture synthesis algorithm is similar to the one proposed by Kopf \cite{KFCODLW07}, it starts with a 2D reference texture
provided by slices extracted from a $\mu{MR}$ or $SR \mu{CT}$ acquisition and by means of an energy optimization process 
described by Kwatra \cite{kwatra:2005:SIGGRAPH}, the method is able to create a 3D texture that resembles the 2D image in every slice.

The texture synthesis optimization has the advantage of creating models, using one or multiple samples 
to constrain the view perpendicular to each axis direction, extra channels can be added to the exemplar, like distance maps 
which are useful to code large textured features \cite{Lefebvre:2006:ATS:1141911.1141921}.
The method also maintains the global statistics of the sample by using a histogram matching approach \cite{ROLLAND2000}.
The following section describes our method to synthesize realistic 3D virtual images from slices of $\mu{MR}$, $SR \mu{CT}$ acquisitions.
In section \ref{sec:ResultsAndEvaluation}, we present the generated virtual images. 
The accuracy of the synthetic images is assessed and discussed by comparing statistical and morphological parameters computed from the virtual and the real images. 