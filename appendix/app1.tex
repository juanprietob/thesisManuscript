% ********** Appendix 1 **********
\chapter{Principal Component Analysis}
\label{sec:apendixPCA}

Principal Component Analysis (PCA) is a way of identifying patterns in the data.
PCA is able to detect the principal directions of variation and
is a powerful tool for analysing data that lives in high dimensional spaces.
Common uses of PCA are to produce compress or compact datasets
by finding and removing redundant information. 
After the compression, any of the points in the original dataset can be represented
using a small number of parameters.

The following sections explains how to use PCA. 

\section{Computing PCA}

The mean of the data must be substracted from each dimension for PCA to work properly.  
By doing this, the data set has mean zero. 

\subsection{Compute the mean}

\begin{equation}
 \bar{x} = \frac{1}{s} \sum^s_{i=1} x_i
\end{equation}
$x$ is a dataset where each column represents an object from the population.  
$s$ is the number of objects in $x$, 
$\bar{x}$ is a vector with the mean of the data calculated for each dimension.

\subsection{Compute the covariance of the data}

\begin{equation}
 cov(X, Y) = \frac{1}{s-1} \sum^s_{i=1} (X_i - \bar{X})(Y_i - \bar{Y})
 \label{equ:covariance}
\end{equation}

\begin{equation}
 C = C_{ab} = cov(Row(x, a), Row(x, b))
 \label{equ:covarianceMatrix}
\end{equation}

The covariance matrix $C$ \ref{equ:covarianceMatrix} is a square matrix that contains at each position the covariance \ref{equ:covariance} between 
the $a$ and $b$ dimension of the dataset $x$. Notice that when $a = b$ we get the regular variance.
$Row(x, a)$ retrieves row $a$ from the data. 

\subsection{Perform eigenanalysis on the covariance}

\begin{equation}
 C = Q \Lambda Q^{-1}
 \label{equ:eigenDecomposition}
\end{equation}
Equation \ref{equ:eigenDecomposition} shows how to perform an eigenanalysis
where $Q$ is a square matrix whose $i^{th}$ column is the eigenvector $q_i$
of $C$ and $\Lambda$ is the diagonal matrix whose diagonal elements are the 
corresponding eigenvalues $\lambda_i$, i.e., $\Lambda_{ii}=\lambda_i$.
All eigenvectors must satisfy the following:

\begin{equation}
 C q_i = \lambda_i q_i 
 \label{equ:eigenAnalysis}
\end{equation}
where $\lambda_i$ is the eigenvalue associated to the eigenvector $q_i$. 
The eigenvectors have unit length, and the eigenvalues 
are sorted so that $\lambda_i \geq \lambda_{i+1}$.

\subsection{Compute the total variance}

\begin{equation}
 V_t = \sum_t \lambda_i
 \label{equ:totalVariance}
\end{equation}

Each eigenvalue gives the variance of the data about the mean in the
direction of the corresponding eigenvector. 

\subsection{Use a number of eigenvalues}

\begin{equation}
 \sum_{i=1}^t \lambda_i \geq f_v V_t
 \label{equ:chooseVectors}
\end{equation}

To represent the data with the most significative information a number of eigenvectors $P$ is chosen
by using their corresponding eigenvalues. 
As show in equation \ref{equ:chooseVectors}, $f_v \in [0, 1]$ defines the proportion of the total variation 
in the data that is going to be explained. 
For example: in order to explain $95\%$ of the information, $f_v = 0.95$, this means that
the first $n$ eigenvectors are used whose eigenvalues add up to $0.95 V$.

\section{Producing a compact data set}

To produce the compressed data set, a matrix $P_t =(p_1 | p_2 | ... | p_t)$ of the eigenvectors chosen to describe the dataset is used, as shown in equation \ref{equ:chooseVectors}.
The matrix $P_t$ is multiplied by the adjusted data set, yielding $x_c$, a compact representation of the data, as shown in equation \ref{equ:compressData}.

\begin{equation}
 x_c = P_t^T (x - \bar{x}) 
 \label{equ:compressData}
\end{equation}

\section{Recovering the original data}

In some situations such as compression, getting the original data back is important. 
When a reduced number of eigenvectors is used, 
the transformation to get the original data, looses some information.
Notice that the only way to get the original data back is by using all the eigenvectors.

\begin{equation}
 (x - \bar{x}) = (P_{t=n}^T)^{-1} x_c
 \label{equ:dataBack0}
\end{equation}

In the case where all the eigenvectors are used, it turns out that the inverse of $P_n$ is equal to the transpose, 
so $(P_{t}^T)^{-1} = P_t$.
This step also applies even when a set of eigenvectors is used, but as mentioned before
some data is lost during the final transformation, as shown in Equation \ref{equ:dataBack1}

\begin{equation}
 x = P_{n} x_c + \bar{x}
 \label{equ:dataBack1}
\end{equation}


% ********** End of appendix **********
